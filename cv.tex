%------------------------------------
% Dario Taraborelli
% Typesetting your academic CV in LaTeX
%
% URL: http://nitens.org/taraborelli/cvtex
% DISCLAIMER: This template is provided for free and without any guarantee 
% that it will correctly compile on your system if you have a non-standard  
% configuration.
% Some rights reserved: http://creativecommons.org/licenses/by-sa/3.0/
%------------------------------------

%!TEX TS-program = xelatex
%!TEX encoding = UTF-8 Unicode

\documentclass[10pt, a4paper]{article}
\usepackage{fontspec} 

% DOCUMENT LAYOUT
\usepackage{geometry} 

\geometry{a4paper, textwidth=6in, textheight=8.5in, marginparsep=7pt, marginparwidth=.6in}
\setlength\parindent{0in}

% FONTS
\usepackage{xunicode}
\usepackage{xltxtra}
 \defaultfontfeatures{Mapping=tex-text} % converts LaTeX specials (``quotes'' --- dashes etc.) to unicode
\setromanfont [Ligatures={Common}, BoldFont={Fontin Bold},
ItalicFont={Fontin Italic}]{Fira Sans}
\setsansfont [Ligatures={Common}, BoldFont={Fontin Sans Bold}, ItalicFont={Fontin Sans Italic}]{Fira Sans}
\setmonofont{Fira Mono} 
% ---- CUSTOM AMPERSAND
\newcommand{\amper}{{\fontspec[Scale=.95]{Fontin}\selectfont\itshape\&}}
% ---- MARGIN YEARS
\usepackage{marginnote}
\newcommand{\years}[1]{\marginnote{\scriptsize #1}}
\renewcommand*{\raggedleftmarginnote}{}
\setlength{\marginparsep}{7pt}
\reversemarginpar

% HEADINGS
\usepackage{sectsty} 
\usepackage[normalem]{ulem} 
\sectionfont{\rmfamily\mdseries\upshape\Large}
\subsectionfont{\rmfamily\bfseries\upshape\normalsize} 
\subsubsectionfont{\rmfamily\mdseries\upshape\normalsize} 

\usepackage{fancyhdr}
\pagestyle{fancy}
\fancyhead{} % clear all header fields
\renewcommand{\headrulewidth}{0pt} % no line in header area
\fancyfoot{} % clear all footer fields
\fancyfoot[LE,RO]{}           % page number in "outer" position of footer line
\fancyfoot[RE,LO]{
email: \href{mailto:sebastian.berndt@gmail.com}{sebastian.berndt@gmail.com}\\
\textsc{url}: \href{http://seberndt.github.io/}{http://seberndt.github.io/}\\ 
} % other info in "inner" position of footer line



% PDF SETUP
% ---- FILL IN HERE THE DOC TITLE AND AUTHOR
\PassOptionsToPackage{hyphens}{url}
\usepackage[bookmarks, colorlinks, breaklinks, pdftitle={Sebastian
  Berndt - vita},pdfauthor={Sebastian Berndt}]{hyperref}  
\hypersetup{linkcolor=blue,citecolor=blue,filecolor=black,urlcolor=blue} 


% DOCUMENT
\begin{document}
{\LARGE Sebastian Berndt}\\[2mm]
\hrule
\vspace{2mm}
{\large Research Areas:}
 approximation algorithms, \textsc{FPT} algorithms,  cryptography

{\large Publications:} AAAI, APPROX, CCS, ESA,  EUROCRYPT, SEA, \ldots 
~(\hyperref[publications]{Link})

{\large Teaching:} Algorithms and Datastructures, Algorithm Design, IT-Security, Coding Theory
(\hyperref[teaching]{Link})

{\large Education:} BSc, MSc, Ph.D.
(\hyperref[education]{Link})\\[-1mm]
\hrule
\vfill




%%\hrule
% \section*{Current position}


%%\hrule

%%\hrule
% \section*{Appointments held}
% \noindent
% \years{2012-}University of Lübeck

%\hrule
\section*{Education}
\label{education}
\years{2010}\textsc{BSc} in Computer Science, Kiel University\\
\years{2012}\textsc{MSc} in Computer Science, Kiel University\\
\years{2018}Ph.D.~in Computer Science, "New Results on Feasibilities and
Limitations of Provable Secure Steganography", Advisor: Prof.~Dr.~Maciej
Liśkiewicz (summa cum laude)\\

\section*{Employment}
\years{2012--2017}Research Associate, Ph.D.~Student, Institute for
Theoretical Computer Science (Prof.~Dr.~Rüdiger Reischuk), University of Lübeck\\
\years{2017--}Research Associate,   Department of Computer
Science (Prof.~Dr.~Klaus Jansen), Kiel University\\

\section*{Awards}
\years{2016}Best Student Paper Award for "Provable Secure Universal
Steganography of Optimal Rate"\\
\years{2016}Third place in the tracks »sequential exact
solver« and »parallel heuristic solver« in the first \emph{PACE}
challenge on parameterized algorithms\\
\years{2017}Third place in »Track A: Treewidth« in the second \emph{PACE}
challenge on parameterized algorithms\\
\years{2018}Best Student Paper Award for "Practical Access to Dynamic Programming on Tree Decompositions"\\

\section*{Talks}
\years{2015a}"Learnability does not imply Secure
Steganography", Nordic Complexity Workshop\\
\years{2015b}"Fully Dynamic Bin Packing Revisited",
 \href{http://www.birs.ca/events/2015/5-day-workshops/15w5118}{Approximation
   Algorithms and Parameterized Complexity}\\
\years{2016a}"Computing tree decompostions via SAT solvers",
Kiel University\\
\years{2016b}"On the Relation between Steganography and Cryptography",
Information Security Seminar, Queensland University of Technology\\
\years{2017}"The PACE challenge: practical algorithms for tree width",
Universidad de Chile\\
\years{2018}"Computing Tree Width: Theory and Practice", University of Bergen



% \vspace{-.1cm}
% \section*{Interests}
% Kayaking, Improvisational Theatre, Hiking


%\hrule
\section*{Publications}
\label{publications}

\subsection*{Conference Proceedings}
% \years{2015}Berndt, Sebastian and Jansen, Klaus and Klein, Kim-Manuel
% (2015),\\ "Fully Dynamic Bin Packing Revisited", \emph{APPROX/RANDOM
%   2015}\\
\years{2016a}Berndt, Sebastian and Reischuk, Rüdiger (2016),\\
"Steganography Based on Pattern Languages", \emph{LATA 2016}\\
\years{2016b}Berndt, Sebastian and Liśkiewicz, Maciej (2016),\\
"Provable Secure Universal Steganography of Optimal Rate", \emph{ACM
  IH{\&}MMSEC 2016}\\
\textbf{Awarded Best Student Paper}\\
\years{2016c}Berndt, Sebastian and Liśkiewicz, Maciej (2016),\\
"Hard Communication Channels for Steganography", \emph{ISAAC 2016}\\
\years{2017a}Berndt, Sebastian and Liśkiewicz, Maciej and Lutter, Matthias
and Reischuk, Rüdiger (2017),\\
"Learning Residual Alternating Automata", \emph{AAAI 2017}\\
\years{2017b}Bannach, Max and Berndt, Sebastian and Ehlers, Thorsten (2017),\\
"Jdrasil: A Modular Library for Computing Tree Decompositions",
\emph{SEA 2017}\\
\years{2017c}Berndt, Sebastian and Liśkiewicz, Maciej (2017),\\
"Algorithm Substitution Attacks from a Steganographic Perspective",
\emph{CCS 2017}\\
\years{2018a} Berndt, Sebastian and Liśkiewicz, Maciej (2018),\\
"On the Gold Standard for Security of Universal Steganography", \emph{EUROCRYPT
  2018}\\
\years{2018b} Berndt, Sebastian (2018),\\
"Computing Tree Width: From Theory to Practice and Back", \emph{CIE 2018}\\
\years{2018c} Berndt, Sebastian and Klein, Kim-Manuel (2018),\\
"Using Structural Properties for Integer Programs", \emph{CIE 2018}\\
\years{2018d} Bannach, Max and Berndt, Sebastian  (2018),\\
"Practical Access to Dynamic Programming on Tree Decompositions", \emph{ESA
  2018}\\
\textbf{Awarded Best Student Paper (Track B)}\\
\years{2018e} Bannach, Max and Berndt, Sebastian and Ehlers, Thorsten and
Nowotka, Dirk (2018),\\
"SAT-Encodings of Tree Decompositions", \emph{SAT COMPETITION 2018}

\subsection*{Journals}
\years{2018} Berndt, Sebastian and Klein, Kim-Manuel and Jansen, Klaus (2018),\\
"Fully Dynamic Bin Packing Revisited", \emph{Math. Program. 2018} (accepted)\\
preliminary version was presented at \emph{APPROX/RANDOM 2015}






\section*{Teaching}
\label{teaching}
- Teaching Assistant for "Algorithm Design" in 2012, 2013, 2014, 2015, and 2016
teaching tutorials and some of the lectures (Lübeck)\\
- Teaching Assistant for "Introduction to IT Security and Reliability" in 2012,
2013, 2014, 2015, and 2016 teaching tutorials and some of the lectures (Lübeck)\\
- Teaching Assistant for "Coding and Security" in 2013, 2014, 2015, and 2016 teaching tutorials and some of the lectures (Lübeck)\\
- Lecturer for  "Presentation and Documentation" in 2015 teaching four lectures (Lübeck)\\
- Teaching Assistant for "Introduction to Operations Research" in 2017 and 2018
teaching tutorials (Kiel)\\
- Teaching Assistant for "Algorithms and Datastructures" in 2018
teaching tutorials (Kiel)\\
- Lecturer for "Online Algorithms" in 2018 teaching and designing the lectures (Kiel)

% \hrule
%\vspace{1cm}

%\hrulefill
\newpage
\subsection*{Supervised Theses}
\years{2015a}Bachelor Thesis on "Lower Bounds in Online Bin Packing
Models"\\
\years{2015b}Bachelor Thesis on "Secure Multiparty Computations in
Bitcoin"\\
\years{2015c}Bachelor Thesis on "Development and Examination of a Huffman-coding
based Stegosystem"\\
\years{2018a}Bachelor Thesis on "Mobility 4.0 - Optimizing Vehicle Planning by
Scheduling Algorithms"\\
\years{2018b}Bachelor Thesis on "Sensitivity Analysis with the Steinitz Lemma"\\


\section*{Extracurricular Activities}
\years{2012--2015}Received the ``\emph{Teaching Certificate II}'' by taking
more than 10 courses in e.g. team leading, presentation techniques and
others (\href{https://www.uni-luebeck.de/universitaet/einrichtungen/dozierenden-service-center/hochschuldidaktik/zertifikatsprogramm.html}{Link})\\
\years{2016}Organizing Commitee of \emph{Creative Mathematical Sciences
  Communication} (\href{http://www.tcs.uni-luebeck.de/cmsc/}{Link})\\
\years{2016}Taught a week-long summer course on algorithms to a group of
pupils from age 14 to 17 based on \emph{Computer Science Unplugged}
(\href{https://www.lias.uni-luebeck.de/veranstaltungen/isc/isc16.html}{Link})\\
\years{2016}Developed the tool \emph{Jdrasil} to compute tree
decompositions (\href{https://github.com/maxbannach/Jdrasil}{Link})\\
\years{2018}Taught a day-long course on algorithmics in the context of the
"Girls' Day" for female pupils from age 14 to 15
(\href{https://www.inf.uni-kiel.de/de/aktuelles/girls-day-am-insitut-fuer-informatik}{Link})\\
\years{2017}Co-wrote a grant proposal on parameterized scheduling problems\\
\years{2018}Taught four lectures of one hour to a group of pupils
(\href{https://www.inf.uni-kiel.de/de/informatik-schule/schnupperstudium/bilder-schnupperstudium/programm-2018}{Link})\\
\years{2018}Co-organized the annual "day of business informatics" (\href{http://www.kn-online.de/Kiel/Tag-der-Wirtschaftsinformatik-Ein-extrem-wichtiges-Studium-in-Kiel}{Link})

\section*{Academic Service}
- I was an external reviewer for the following conferences: \emph{STOC},
\emph{ESA}, \emph{ICALP}, \emph{STACS}, \emph{IPDPS}, \emph{ALT}, \emph{WG},
\emph{LATIN}, \emph{WAOA}, 
 \emph{SOFSEM}, \emph{CIE}, \emph{OPTA}\\
- I was a reviewer for the following journals: \emph{Algorithmica}, \emph{JAIR}, \emph{JEA}





\vspace{-.05cm}
\vfill{}
\begin{center}

{\scriptsize  Last updated: \today\- •\- Typeset in \href{http://nitens.org/taraborelli/cvtex}{
\fontspec{Times New Roman}  \XeLaTeX}\\
% ---- FILL IN THE FULL URL TO YOUR CV HERE
}
\end{center}

\end{document}

% Local Variables:
% TeX-engine: xetex
% End:
