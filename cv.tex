%------------------------------------
% Dario Taraborelli
% Typesetting your academic CV in LaTeX
%
% URL: http://nitens.org/taraborelli/cvtex
% DISCLAIMER: This template is provided for free and without any guarantee 
% that it will correctly compile on your system if you have a non-standard  
% configuration.
% Some rights reserved: http://creativecommons.org/licenses/by-sa/3.0/
%------------------------------------

%!TEX TS-program = xelatex
%!TEX encoding = UTF-8 Unicode

\documentclass[10pt, a4paper]{article}
\usepackage{fontspec} 

% DOCUMENT LAYOUT
\usepackage{geometry} 

\geometry{a4paper, textwidth=6in, textheight=8.5in, marginparsep=7pt, marginparwidth=.6in}
\setlength\parindent{0in}

% FONTS
\usepackage{xunicode}
\usepackage{xltxtra}
 \defaultfontfeatures{Mapping=tex-text} % converts LaTeX specials (``quotes'' --- dashes etc.) to unicode
\setromanfont [Ligatures={Common}, BoldFont={Fontin Bold},
ItalicFont={Fontin Italic}]{Fira Sans}
\setsansfont [Ligatures={Common}, BoldFont={Fontin Sans Bold}, ItalicFont={Fontin Sans Italic}]{Fira Sans}
\setmonofont{Fira Mono} 
% ---- CUSTOM AMPERSAND
\newcommand{\amper}{{\fontspec[Scale=.95]{Fontin}\selectfont\itshape\&}}
% ---- MARGIN YEARS
\usepackage{marginnote}
\newcommand{\years}[1]{\marginnote{\scriptsize #1}}
\renewcommand*{\raggedleftmarginnote}{}
\setlength{\marginparsep}{7pt}
\reversemarginpar

% HEADINGS
\usepackage{sectsty} 
\usepackage[normalem]{ulem} 
\sectionfont{\rmfamily\mdseries\upshape\Large}
\subsectionfont{\rmfamily\bfseries\upshape\normalsize} 
\subsubsectionfont{\rmfamily\mdseries\upshape\normalsize} 

\usepackage{fancyhdr}
\pagestyle{fancy}
\fancyhead{} % clear all header fields
\renewcommand{\headrulewidth}{0pt} % no line in header area
\fancyfoot{} % clear all footer fields
\fancyfoot[LE,RO]{}           % page number in "outer" position of footer line
\fancyfoot[RE,LO]{Phone: \texttt{+49-451-3101-5323}\\
email: \href{mailto:berndt@tcs.uni-luebeck.de}{berndt@tcs.uni-luebeck.de}\\
\textsc{url}: \href{http://www.tcs.uni-luebeck.de/de/mitarbeiter/berndt/}{http://www.tcs.uni-luebeck.de/de/mitarbeiter/berndt/}\\ 
} % other info in "inner" position of footer line



% PDF SETUP
% ---- FILL IN HERE THE DOC TITLE AND AUTHOR
\PassOptionsToPackage{hyphens}{url}
\usepackage[bookmarks, colorlinks, breaklinks, pdftitle={Sebastian
  Berndt - vita},pdfauthor={Sebastian Berndt}]{hyperref}  
\hypersetup{linkcolor=blue,citecolor=blue,filecolor=black,urlcolor=blue} 


% DOCUMENT
\begin{document}
{\LARGE Sebastian Berndt}\\[2mm]
\hrule
\vspace{2mm}
{\large Research Areas:}
steganography, cryptography, approximation algorithms, \textsc{FPT}
algorithms

{\large Publications:} AAAI, APPROX, IH{\&}MMSEC, ISAAC, LATA, SEA
(\hyperref[publications]{Link})

{\large Teaching:} Algorithm Design, IT-Security, Coding Theory
(\hyperref[teaching]{Link})

{\large Education:} BSc, MSc, Ph.\,D.\,Student
(\hyperref[education]{Link})\\[-1mm]
\hrule
\vfill




%%\hrule
% \section*{Current position}


%%\hrule

%%\hrule
% \section*{Appointments held}
% \noindent
% \years{2012-}University of Lübeck

%\hrule
\section*{Education}
\label{education}
\years{2010}\textsc{BSc} in Computer Science, University of Kiel\\
\years{2012}\textsc{MSc} in Computer Science, University of Kiel\\
\years{2012--}Research Associate, Ph.\,D.\,Student, University of Lübeck


%\hrule
\section*{Publications}
\label{publications}
 Rankings are from the 2014 edition of the Computing Research
and Education Association of Australasia Conference Ratings Exercise (CORE 2014),
ranging from A$^{*}$ (exceptional) to C (sound and satisfactory).\\[.4cm]

\years{2015}Berndt, Sebastian and Jansen, Klaus and Klein, Kim-Manuel
(2015),\\ "Fully Dynamic Bin Packing Revisited", \emph{APPROX/RANDOM
  2015}, \href{http://portal.core.edu.au/conf-ranks/1451/}{Rating: A}\\
\years{2016a}Berndt, Sebastian and Reischuk, Rüdiger (2016),\\
"Steganography Based on Pattern Languages", \emph{LATA 2016},
\href{http://portal.core.edu.au/conf-ranks/1115/}{Rating: C}\\
\years{2016b}Berndt, Sebastian and Liśkiewicz, Maciej (2016),\\
"Provable Secure Universal Steganography of Optimal Rate", \emph{ACM
  IH{\&}MMSEC 2016}, \href{http://portal.core.edu.au/conf-ranks/1493/}{Rating: C}\\
\textbf{Awarded Best Student Paper}\\
\years{2016c}Berndt, Sebastian and Liśkiewicz, Maciej (2016),\\
"Hard Communication Channels for Steganography", \emph{ISAAC 2016},
\href{http://portal.core.edu.au/conf-ranks/1349/}{Rating: A}\\
\years{2017a}Berndt, Sebastian and Liśkiewicz, Maciej and Lutter, Matthias
and Reischuk, Rüdiger (2017),\\
"Learning Residual Alternating Automata", \emph{AAAI 2017},
\href{http://portal.core.edu.au/conf-ranks/1629/}{Rating: A$^{*}$}\\
\years{2017b}Bannach, Max and Berndt, Sebastian and Ehlers, Thorsten (2017),\\
"Jdrasil: A Modular Library for Computing Tree Decompositions",
\emph{SEA 2017}, \href{http://portal.core.edu.au/conf-ranks/1378/}{Rating: B}




\section*{Talks}
\years{2015a}"Learnability does not imply Secure
Steganography",\\ Nordic Complexity Workshop\\
\years{2015b}"Fully Dynamic Bin Packing Revisited",\\
 \href{http://www.birs.ca/events/2015/5-day-workshops/15w5118}{Approximation
   Algorithms and Parameterized Complexity}\\
\years{2016a}"Berechnung von Baumzerlegungen mit SAT-Solvern",\\
University of Kiel\\
\years{2016b}"On the Relation between Steganography and Cryptography",\\
Information Security Seminar, Queensland University of Technology



\section*{Teaching}
\label{teaching}
\years{2012a}Exercises on "Algorithm Design"\\
\years{2012b}Exercises on "Introduction to IT Security and Reliability"\\
\years{2013a}Exercises on "Coding and Security"\\
\years{2013b}Exercises on "Algorithm Design"\\
\years{2013c}Exercises on "Introduction to IT Security and Reliability" \\
\years{2014a}Exercises on "Coding and Security" \\
\years{2014b}Exercises on "Algorithm Design" \\
\years{2014c}Exercises on "Introduction to IT Security and Reliability"\\
\years{2015a}Exercises on "Coding and Security"\\
\years{2015b}Exercises on "Algorithm Design"  \\
\years{2015c}Lectures and Exercises on "Introduction to IT Security and Reliability" \\
\years{2015d}Lectures on "Presentation and Documentation"\\
\years{2016a}Exercises on "Coding and Security"\\
\years{2016b}Exercises on "Algorithm Design"\\
\years{2016c}Lectures and Exercises on "Introduction to IT Security and Reliability"
%\hrule
%\vspace{1cm}

%\hrulefill

\subsection*{Theses}
I was involved in the following theses, but was not formally one of
the supervisors.\\[.4cm]
\years{2015a}Bachelor Thesis on "Lower Bounds in Online Bin Packing
Models"\\
\years{2015b}Bachelor Thesis on "Secure Multiparty Computations in
Bitcoin"\\
\years{2015c}Bachelor Thesis on "Development and Examination of a Huffman-coding based Stegosystem"

\section*{Extracurricular Activities}
\years{2012--2015}Received the ``\emph{Teaching Certificate I}'' by taking
more than 10 courses in e.g. team leading, presentation techniques and
others (\href{https://www.uni-luebeck.de/universitaet/einrichtungen/dozierenden-service-center/hochschuldidaktik/zertifikatsprogramm.html}{Link})\\
\years{2016}Organizing Commitee of \emph{Creative Mathematical Sciences
  Communication} (\href{http://www.tcs.uni-luebeck.de/cmsc/}{Link})\\
\years{2016}Taught a week-long summer course on algorithms to a group of
pupils from age 14 to 17 based on \emph{Computer Science Unplugged}
(\href{https://www.lias.uni-luebeck.de/veranstaltungen/isc/isc16.html}{Link})\\
\years{2016}Developed the tool \emph{Jdrasil} to compute tree
decompisitions which got the third place in the tracks »sequential exact
solver« and »parallel heuristic solver« in the first \emph{PACE}
challenge on parameterized algorithms
(\href{https://github.com/maxbannach/Jdrasil}{Software},\href{https://pacechallenge.wordpress.com/}{Challenge})




\section*{Awards}
\years{2016}Best Student Paper Award for "Provable Secure Universal
Steganography of Optimal Rate"\\
\years{2016}Third place in the tracks »sequential exact
solver« and »parallel heuristic solver« in the first \emph{PACE}
challenge on parameterized algorithms

\vspace{-.1cm}
\section*{Interests}
Kayaking, Improvisational Theatre, Hiking

\vspace{-.05cm}
\vfill{}
\begin{center}

{\scriptsize  Last updated: \today\- •\- Typeset in \href{http://nitens.org/taraborelli/cvtex}{
\fontspec{Times New Roman} }\\
% ---- FILL IN THE FULL URL TO YOUR CV HERE
\href{http://seberndt.github.io/cv.pdf}{http://seberndt.github.io/cv.pdf}}
\end{center}

\end{document}
