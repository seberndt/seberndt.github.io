\begin{rubric}{Repräsentative Publikationen}
  \entry*[CCS 2020] In der Arbeit \emph{SNI-in-the-head: Protecting MPC-in-the-head Protocols against Side-channel Analysis} haben wir das MPC-in-the-head Entwurfsmuster zur Konstruktion von Zero-Knowledge Beweisen untersucht, welches inzwischen in zahlreichen Anwendungen benutzt wird, unter anderem im Post-Quantum-Signaturverfahren \textsf{Picnic}. Wir waren zunächst in der Lage, einen konkreten Seitenkanalangriff auf nahezu alle Verfahren, die dieses Entwurfsmuster nutzen, zu finden. Als Gegenmaßnahme konnten wir das Verfahren so erweitern, dass eine große Klasse von Seitenkanalangriffen beweisbar ausgeschlossen werden kann. Dies ist das erste Verfahren, welches einen Schutz gegen Seitenkanalangriffe bereits auf der Protokoll-Ebene umsetzt.

  \entry*[ESA 2019] In der Arbeit \emph{Online Bin Covering with Limited Migration} haben wir uns mit einem Semi-Online-Szenario auseinander gesetzt. Hierbei erscheinen Objekte über die Zeit, dürfen aber, im Gegensatz zum reinen Online-Szenario, auch für gewisse Kosten umgepackt werden. Wir haben uns mit dem Problem \textsc{Bin Covering} beschäftigt, welches als duales Problem zu \textsc{Bin Packing} das Ziel hat, möglichst viele Behälter zu füllen. Hierbei haben wir alle vier möglichen Fälle (mit und ohne Löschen von Objekten und mit und ohne Amortisierung) untersucht und jeweils Algorithmen mit sehr beschränkten Umpackungskosten entwickelt. Weiterhin haben wir für alle vier Fälle zeigen können, dass diese Umpackungskosten asymptotisch optimal sind, also kein Algorithmus niedrigere Kosten haben kann.


  \entry*[EUROCRYPT 2018] In der Arbeit \emph{On the Gold Standard for Security of Universal Steganography} haben wir uns mit asymmetrischen Verfahren der Steganographie beschäftigt. Backes und Cachin hatten gezeigt, dass es ein asymmetrisches Stegosystem gibt, welches für viele Kommunikationskanäle einen relativ schwachen Sicherheitsbegriff erfüllt. In einer Nachfolgearbeit konnte Hopper zeigen, dass auch ein stärkerer Sicherheitsbegriff erfüllt werden kann, jedoch muss das Stegosystem hier auf den Kanal zugeschnitten werden. In unserer Arbeit konnten wir zeigen, dass dieser stärkere Sicherheitsbegriff auch durch ein universelles Stegosystem für viele Kanäle erreicht werden kann. Zusätzlich konnten wir zeigen, dass dies bestmöglich ist, indem wir diesen starken Sicherheitsbegriff für andere Kanäle ausschließen konnten. Dies ist ein starker Kontrast zu Kryptosystemen, bei denen seit langem bekannt ist, wie ein analoger starker Sicherheitsbegriff gewährleistet werden kann. 

  \entry*[ESA 2018] In der Arbeit \emph{Practical Access to Dynamic Programming on Tree Decompositions} haben wir uns mit dynamischer Programmierung auf sogenannten Baumzerlegungen beschäftigt. Diese Baumzerlegungen geben die Ähnlichkeit eines gegegebenen Graphens zu einem Baum an und haben viele Anwendungen im Bereich der parametrisierten Komplexität. In dieser Arbeit haben wir ein Tool entwickelt, welches es erlaubt, solche dynamischen Programme sehr einfach zu beschreiben und direkt in Verbindung mit state-of-the-art-Solvern für Baumzerlegungen zu benutzen. Weiterhin waren wir in der Lage, ein Fragment der MSO-Logik zu formulieren, welches ausdrucksstark genug ist, um nahezu alle relevanten Fragestellungen zu Graphen mit beschränkter Baumweiter zu formalisieren. Mithilfe unserers Tools können diese Fragestellungen nun direkt mit Algorithmen zur Baumzerlegung kombiniert werden und die so entstehenden Algorithmen sind kompetitiv mit vielen spezialisierten state-of-the-art Verfahren. 


  % \entry*[CCS 2017] In der Arbeit \emph{Algorithm Substitution Attacks from a Steganographic Perspective} haben wir uns mit sogenannten Substitutions-Angriffen beschäftigt, bei denen einzelne Algorithmen eines kryptographischem Primitives durch eine modifizierte Version ausgetauscht werden, die von eienm Angreifer benutzt werden kann. Diese Angriffe  wurden in der Mitte der 90er Jahre untersucht und haben seit den NSA-Enthüllungen durch Edward Snowden einen wichtigen Platz in der IT-Sicherheits-Forschung eingenommen, da solche Angriffe es erlauben, typische großskalige Angriffe zu modellieren und die Fähigkeit solch mächtiger Angreifer zu untersuchen. Wir waren in dieser Arbeit in der Lage zu zeigen, dass dieses Modell eine sehr natürliche Darstellung in der Steganographie hat. Durch diesen engen Zusammenhang waren wir in der Lage, viele aktuelle Forschungsergebnisse im Bereich der Substitutions-Angriffe auf steganographische Fragestellungen zurückzuführen. Weiterhin waren wir in der Lage, obere Schranken für die Menge an Informationen zu finden, die durch solch einen Angriff an einen Angreifer kommuniziert werden kann.

  \entry*[AAI 2017] In der Arbeit \emph{Learning Residual Alternating Automata} haben wir uns mit dem aktiven Lernen von regulären Sprachen beschäftigt. Der erste entsprechende Algorithmus von Angluin war in der Lage, solche Sprachen als endliche deterministische Automaten darzustellen. Dieser Algorithmus wurde von Bollig, Habermehl, Kern und Leucker verallgemeinert, so dass er auch nichtdeterministische Automaten lernen kann, welche bewiesenermaßen exponentiell kleiner sein können. Angluin, Eisenstat und Fisman wiederum konnte dies auf alternierende Automaten übertragen, die wiederum exponentiell kleiner als nichtdeterministische Automaten sein können. Ein zentraler Begriff in all diesen Algorithmen ist der Begriff der Residualität, der es erlaubt, den einzelnen Zuständen der Automaten eine sinnvolle Semantik zuzuordnen. Angluin, Eisenstat und Fisman hatten in ihrer Arbeit die Vermutung aufgestellt, dass ihr Algorithmus ebenfalls solche residualen Automaten lernt. Wir konnten mittels eines komplexen Gegenbeispiels beweisen, dass diese Vermutung nicht der Wahrheit entspricht. Weiterhin waren wir in der Lage, einen Algorithmus zu konstruieren, der beweisbar nur residuale Automaten lernt und zu zeigen, dass diese auch weiterhin exponentiell kleiner als nichtdeterministische Automaten sein können. 

\end{rubric}