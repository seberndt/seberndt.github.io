%------------------------------------
% Dario Taraborelli
% Typesetting your academic CV in LaTeX
%
% URL: http://nitens.org/taraborelli/cvtex
% DISCLAIMER: This template is provided for free and without any guarantee 
% that it will correctly compile on your system if you have a non-standard  
% configuration.
% Some rights reserved: http://creativecommons.org/licenses/by-sa/3.0/
%------------------------------------

%!TEX TS-program = xelatex
%!TEX encoding = UTF-8 Unicode

\documentclass[10pt, a4paper]{article}
\usepackage{fontspec} 

% DOCUMENT LAYOUT
\usepackage{geometry} 

\geometry{a4paper, textwidth=6in, textheight=8.5in, marginparsep=7pt, marginparwidth=.6in}
\setlength\parindent{0in}

% FONTS
\usepackage{xunicode}
\usepackage{xltxtra}
 \defaultfontfeatures{Mapping=tex-text} % converts LaTeX specials (``quotes'' --- dashes etc.) to unicode
\setromanfont [Ligatures={Common}, BoldFont={Fontin Bold},
ItalicFont={Fontin Italic}]{Fira Sans}
\setsansfont [Ligatures={Common}, BoldFont={Fontin Sans Bold}, ItalicFont={Fontin Sans Italic}]{Fira Sans}
\setmonofont{Fira Mono} 
% ---- CUSTOM AMPERSAND
\newcommand{\amper}{{\fontspec[Scale=.95]{Fontin}\selectfont\itshape\&}}
% ---- MARGIN YEARS
\usepackage{marginnote}
\newcommand{\years}[1]{\marginnote{\scriptsize #1}}
\renewcommand*{\raggedleftmarginnote}{}
\setlength{\marginparsep}{7pt}
\reversemarginpar

% HEADINGS
\usepackage{sectsty} 
\usepackage[normalem]{ulem} 
\sectionfont{\rmfamily\mdseries\upshape\Large}
\subsectionfont{\rmfamily\bfseries\upshape\normalsize} 
\subsubsectionfont{\rmfamily\mdseries\upshape\normalsize} 

\usepackage{fancyhdr}
\pagestyle{fancy}
\fancyhead{} % clear all header fields
\renewcommand{\headrulewidth}{0pt} % no line in header area
\fancyfoot{} % clear all footer fields
\fancyfoot[LE,RO]{}           % page number in "outer" position of footer line
\fancyfoot[RE,LO]{Phone: \texttt{+49-451-3101-5323}\\
email: \href{mailto:berndt@tcs.uni-luebeck.de}{berndt@tcs.uni-luebeck.de}\\
\textsc{url}: \href{http://www.tcs.uni-luebeck.de/de/mitarbeiter/berndt/}{http://www.tcs.uni-luebeck.de/de/mitarbeiter/berndt/}\\ 
} % other info in "inner" position of footer line



% PDF SETUP
% ---- FILL IN HERE THE DOC TITLE AND AUTHOR
\PassOptionsToPackage{hyphens}{url}
\usepackage[bookmarks, colorlinks, breaklinks, pdftitle={Sebastian
  Berndt - vita},pdfauthor={Sebastian Berndt}]{hyperref}  
\hypersetup{linkcolor=blue,citecolor=blue,filecolor=black,urlcolor=blue} 


% DOCUMENT
\begin{document}
{\LARGE Sebastian Berndt}\\[2mm]
\hrule
\vspace{2mm}
{\large Research Areas:}
steganography, cryptography, approximation algorithms, \textsc{FPT}
algorithms

{\large Publications:} APPROX, LATA, IH{\&}MMSEC
(\hyperref[publications]{Link})

{\large Teaching:} Algorithm Design, IT-Security, Coding Theory
(\hyperref[teaching]{Link})

{\large Education:} B.\,Sc.\,, M.\,Sc.\,, Ph.\,D.\,Student
(\hyperref[education]{Link})\\[-1mm]
\hrule
\vfill




%%\hrule
% \section*{Current position}


%%\hrule

%%\hrule
% \section*{Appointments held}
% \noindent
% \years{2012-}University of Lübeck

%\hrule
\section*{Education}
\label{education}
\years{2010}\textsc{BSc} in Computer Science, University of Kiel\\
\years{2012}\textsc{MSc} in Computer Science, University of Kiel\\
\years{2012--}Research Associate, Ph.\,D.\,Student, University of Lübeck


%\hrule
\section*{Publications \amper{} talks}
\label{publications}
\years{2015}Berndt, Sebastian and Jansen, Klaus and Klein, Kim-Manuel
(2015),\\ "Fully Dynamic Bin Packing Revisited", \emph{APPROX/RANDOM
  2015}\\
\years{2015}Berndt, Sebastian,\\ "Fully Dynamic Bin Packing Revisited",
 \href{http://www.birs.ca/events/2015/5-day-workshops/15w5118}{Approximation Algorithms and Parameterized Complexity}\\
\years{2016}Berndt, Sebastian and Reischuk, Rüdiger (2016),\\
"Steganography Based on Pattern Languages", \emph{LATA 2016}\\
\years{2016}Berndt, Sebastian and Liśkiewicz, Maciej (2016),\\
"Provable Secure Universal Steganography of Optimal Rate", \emph{ACM
  IH{\&}MMSEC 2016},\\
\textbf{Awarded Best Student Paper}.


\section*{Teaching}
\label{teaching}
\years{2012}Exercises on "Algorithm Design"\\
\years{2012}Exercises on "Introduction to IT Security and Reliability"\\
\years{2013}Exercises on "Coding and Security"\\
\years{2013}Exercises on "Algorithm Design"\\
\years{2013}Exercises on "Introduction to IT Security and Reliability" \\
\years{2014}Exercises on "Coding and Security" \\
\years{2014}Exercises on "Algorithm Design" \\
\years{2014}Exercises on "Introduction to IT Security and Reliability" \\
\years{2015}Exercises on "Coding and Security"\\
\years{2015}Exercises on "Algorithm Design"  \\
\years{2015}Exercises on "Introduction to IT Security and Reliability" \\
\years{2015}Lectures on "Presentation and Documentation"\\
\years{2016}Exercises on "Coding and Security"
%\hrule
%\vspace{1cm}

%\hrulefill


\section*{Extracurricular Activities}
\years{2012--}Received the ``\emph{Teaching Certificate I}'' by taking
more than 10 courses in e.g. team leading, presentation techniques and
others (\href{https://www.uni-luebeck.de/universitaet/einrichtungen/dozierenden-service-center/hochschuldidaktik/zertifikatsprogramm.html}{Link})\\
\years{2016}Organizing Commitee of \emph{Creative Mathematical Sciences
  Communication} (\href{http://www.tcs.uni-luebeck.de/cmsc/}{Link})\\
\years{2016}Taught a course on algorithmics for pupils in the
\emph{summer camp}
(\href{https://www.lias.uni-luebeck.de/veranstaltungen/isc/isc16.html}{Link})\\
\years{2016}Developed the tool \emph{Jdrasil} to compute tree
decompisitions (\href{https://github.com/maxbannach/Jdrasil}{Link})

\section*{Awards}
\years{2016}Best Student Paper Award for "Provable Secure Universal
Steganography of Optimal Rate"

\section*{Interests}
Kayaking, Improvisational Theatre, Hiking


\vfill{}
\begin{center}

{\scriptsize  Last updated: \today\- •\- Typeset in \href{http://nitens.org/taraborelli/cvtex}{
\fontspec{Times New Roman} }\\
% ---- FILL IN THE FULL URL TO YOUR CV HERE
\href{http://nitens.org/taraborelli/cvtex}{http://nitens.org/taraborelli/cvtex}}
\end{center}

\end{document}
